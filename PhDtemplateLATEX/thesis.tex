
% ----------------------------------------------------------------------
%                   LATEX TEMPLATE FOR PhD THESIS
% ----------------------------------------------------------------------

% based on Harish Bhanderi's PhD/MPhil template, then Uni Cambridge
% http://www-h.eng.cam.ac.uk/help/tpl/textprocessing/ThesisStyle/
% corrected and extended in 2007 by Jakob Suckale, then MPI-CBG PhD programme
% and made available through OpenWetWare.org - the free biology wiki


%: Style file for Latex
% Most style definitions are in the external file PhDthesisPSnPDF.
% In this template package, it can be found in ./Latex/Classes/
\documentclass[twoside,11pt]{Latex/Classes/PhDthesisPSnPDF}


%: Macro file for Latex
% Macros help you summarise frequently repeated Latex commands.
% Here, they are placed in an external file /Latex/Macros/MacroFile1.tex
% An macro that you may use frequently is the figuremacro (see introduction.tex)
\include{Latex/Macros/MacroFile1}



%: ----------------------------------------------------------------------
%:                  TITLE PAGE: name, degree,..
% ----------------------------------------------------------------------
% below is to generate the title page with crest and author name

%if output to PDF then put the following in PDF header
\ifpdf  
    \pdfinfo { /Title  (PhD and MPhil Thesis Classes)
               /Creator (TeX)
               /Producer (pdfTeX)
               /Author (Sjoerd Bergmans sbergmans87@gmail.com)
               /CreationDate (D:YYYYMMDDhhmmss)  %format D:YYYYMMDDhhmmss
               /ModDate (D:YYYYMMDDhhmm)
               /Subject (xyz)
               /Keywords (add, your, keywords, here) }
    \pdfcatalog { /PageMode (/UseOutlines)
                  /OpenAction (fitbh)  }
\fi


\title{E-Depot \& Usuability: Scrolling Back Through Time}



% ----------------------------------------------------------------------
% The section below defines www links/email for author and institutions
% They will appear on the title page of the PDF and can be clicked
\ifpdf
  \author{\href{mailto:sbergmans@gmail.com}{Sjoerd Bergmans}}
  \cityofbirth{Born in: 's-Hertogenbosch} % uncomment this if your university requires this
%  % If city of birth is required, also uncomment 2 sections in PhDthesisPSnPDF
%  % Just search for the "city" and you'll find them.
  \collegeordept{\href{http://www.uva.nl/onderwijs/bachelor/bacheloropleidingen/content/culturele-informatiewetenschap/b-culturele-informatiewetenschap.html}{Media \& Information}}
  \university{\href{http://www.uva.nl}{University of Amsterdam}}

  % The crest is a graphics file of the logo of your research institution.
  % Place it in ./0_frontmatter/figures and specify the width
  \crest{\includegraphics[width=10cm]{logo}}
  
% If you are not creating a PDF then use the following. The default is PDF.
\else
  \author{YourName}
%  \cityofbirth{born in XYZ}
  \collegeordept{CollegeOrDept}
  \university{University}
  \crest{\includegraphics[width=4cm]{logo}}
\fi

\renewcommand{\submittedtext}{A research article submitted for the:}
\degree{Humanities Honours Programme}
\degreedate{\today}


% ----------------------------------------------------------------------
       
% turn of those nasty overfull and wunderfull hboxes
\hbadness=10000
\hfuzz=50pt


%: --------------------------------------------------------------
%:                  FRONT MATTER: dedications, abstract,..
% --------------------------------------------------------------

\begin{document}

%\language{english}

% sets line spacing
\renewcommand\baselinestretch{1.2}
\baselineskip=18pt plus1pt


%: ----------------------- generate cover page ------------------------

\maketitle  % command to print the title page with above variables


%: ----------------------- cover page back side ------------------------
% Your research institution may require reviewer names, etc.
% This cover back side is required by Dresden Med Fac; uncomment if needed.

\newpage
\vspace{10mm}
1. Reviewer: Marijn Koolen

\vspace{10mm}
2. Reviewer: 

\vspace{20mm}
Uiterste Inleverdatum:  19th of January 2015

\vspace{20mm}
\hspace{70mm}Signature of approval:



%: ----------------------- abstract ------------------------

% Your institution may have specific regulations if you need an abstract and where it is to be placed in the document. The default here is just after title.


% Thesis Abstract -----------------------------------------------------


%\begin{abstractslong}    %uncommenting this line, gives a different abstract heading
\begin{abstracts}        %this creates the heading for the abstract page

In this paper current archival institutions are criticized for both their (lack of) response and their way of (not) adapting to an increasingly digital world. Many of the efforts that have been undertaken by current archival institutions are proven to be efforts to enforce 'the paper way of doing things' onto a digitizing world, causing a constant friction between the two. The author of this paper tries to take a leap of faith by establishing a revolutionary vision on the future of archival institutions and the act of archiving in general. Old 'paper views', such as expecting records to be of a singular nature, have a limited (active) life-span, a pre-defined form and a rigid archival bond and context are challenged by introducing new concepts such as 'the never-ending and ever-multiplying record'. By re-inventing, renewing and transforming old concepts into new ones, such as 'fluent metadata', the author finds new ways to protect the authenticity and reliability of these modern records over time, as well as enabling them to be part of a new and spectacular end-user experience, which seems to be left out of the current programs focusing on the development and implementation of E-depots in the Netherlands. 

\end{abstracts}
%\end{abstractlongs}


% ---------------------------------------------------------------------- 


% The original template provides and abstractseparate environment, if your institution requires them to be separate. I think it's easier to print the abstract from the complete thesis by restricting printing to the relevant page.
% \begin{abstractseparate}
%   
% Thesis Abstract -----------------------------------------------------


%\begin{abstractslong}    %uncommenting this line, gives a different abstract heading
\begin{abstracts}        %this creates the heading for the abstract page

In this paper current archival institutions are criticized for both their (lack of) response and their way of (not) adapting to an increasingly digital world. Many of the efforts that have been undertaken by current archival institutions are proven to be efforts to enforce 'the paper way of doing things' onto a digitizing world, causing a constant friction between the two. The author of this paper tries to take a leap of faith by establishing a revolutionary vision on the future of archival institutions and the act of archiving in general. Old 'paper views', such as expecting records to be of a singular nature, have a limited (active) life-span, a pre-defined form and a rigid archival bond and context are challenged by introducing new concepts such as 'the never-ending and ever-multiplying record'. By re-inventing, renewing and transforming old concepts into new ones, such as 'fluent metadata', the author finds new ways to protect the authenticity and reliability of these modern records over time, as well as enabling them to be part of a new and spectacular end-user experience, which seems to be left out of the current programs focusing on the development and implementation of E-depots in the Netherlands. 

\end{abstracts}
%\end{abstractlongs}


% ---------------------------------------------------------------------- 

% \end{abstractseparate}


%: ----------------------- tie in front matter ------------------------

\frontmatter
% Thesis Dedictation ---------------------------------------------------

\begin{dedication} %this creates the heading for the dedication page

You may say I'm a dreamer, but I'm not the only one...

\end{dedication}

% ----------------------------------------------------------------------
\include{0_frontmatter/acknowledgement}


%: ----------------------- contents ------------------------

\renewcommand{\contentsname}{Conceptual setup}
\setcounter{secnumdepth}{3} % organisational level that receives a numbers
\setcounter{tocdepth}{3}    % print table of contents for level 3
\tableofcontents            % print the table of contents
% levels are: 0 - chapter, 1 - section, 2 - subsection, 3 - subsection


%: ----------------------- list of figures/tables ------------------------

%\listoffigures	% print list of figures

%\listoftables  % print list of tables


%: ----------------------- glossary ------------------------

% Tie in external source file for definitions: /0_frontmatter/glossary.tex
% Glossary entries can also be defined in the main text. See glossary.tex
%\include{0_frontmatter/glossary} 

%\begin{multicols}{2} % \begin{multicols}{#columns}[header text][space]
%\begin{footnotesize} % scriptsize(7) < footnotesize(8) < small (9) < normal (10)

%\printnomenclature[1.5cm] % [] = distance between entry and description
%\label{nom} % target name for links to glossary

%\end{footnotesize}
%\end{multicols}



%: --------------------------------------------------------------
%:                  MAIN DOCUMENT SECTION
% --------------------------------------------------------------

% the main text starts here with the introduction, 1st chapter,...
\mainmatter

\renewcommand{\chaptername}{} % uncomment to print only "1" not "Chapter 1"


%: ----------------------- subdocuments ------------------------

% Parts of the thesis are included below. Rename the files as required.
% But take care that the paths match. You can also change the order of appearance by moving the include commands.

%\include{1_introduction/introduction}	% background information

\chapter{Introduction}

\chapter{Explaining the Classic Archival Landscape and Institutions}

\section{Classic landscape}
\subsection{Life-cycle landscape}

The current Dutch archival landscape is based upon the lifecycle model. 

\section{Classic Front-Office} % section headings are printed smaller than chapter names
% intro
\subsection{A limited, well-known public}

The main focus of the front office is on serving the main customer: the public. This public used to exist mainly out of two groups: historically oriented researchers and genealogists.

\section{Classic Back-Office}
\subsection{Checking up on friends}

\chapter{The current vision(s) of the future, and why they are lacking}
\subsection{Werkgroep Voorbereiding Implementatie E-Depot}
\subsubsection{Handboek rollen - taken - verantwoordelijkheden en competenties}
\subsubsection{Opleidingsplan}
\subsubsection{Toepassingsprofiel Lokale Overheden}
\subsubsection{Handboek werkprocessen}
\subsubsection{Enterprisearchitectuur}
\subsection{Archiefcoalitie Digitale Duurzaamheid}
\subsubsection{Voorstel gemeenschappelijke E-depotvoorzieningen en services}
\subsubsection{Rapport E-depot}
\subsection{Consortium Regionale Historische Centra}
\subsubsection{Visiedocument 2009-2014}
\subsection{Landelijk Overleg van Provinciale Archiefinspecteurs}
\subsubsection{ED3}
\subsection{Current Record Professionals}
\subsubsection{McLeod: "records professionals may be part of the problem"} 
-McLeod: their demands may be unrealistic or too constraining

\chapter{Challenging the future: Reuniting the Front- \& Back Office and re-establishing contact with the general public via the E-Depot (evolution vs revolution)}

\section{E-Depot: requirements for digital Front-Office tasks}
\subsection{A new focus on usability: building a digital community with the unknown public}
\subsubsection{McLeod: It will be automatic, ubiquitous and intrinsic without being a burden}
\subsubsection{McLeod: Embed information management in human behavior using easy solutions and
simple processes}
\subsection{A new task: becoming a data mediator}
\section{E-Depot: requirements for digital Back-Office tasks}
\subsection{E-Depot developmental phase}
\subsubsection{Multidisciplinarity}
-McLeod: problem with record managers:
* isolation of records professionals ? going it alone and not involving others early enough in process
* Assuming IT have same knowledge/understanding of what is meant by ERM as yourself
* leave it solely to [records managers] as they will develop an idealised version for the idealised user.
McLeod: there is no ?one-size-fits-all? approach to successful ERM
\subsubsection{McLeod: RM principles need to be used at the systems design / conception phase}
\subsection{Towards a new level of cooperation: learning from and challenging friends}
\subsubsection{Samen werken of samenwerken? Jantine Ploeg}
\subsection{Introducing: fluent records}
\subsubsection{McLeod: records management principles are fundamentally sound and appear to be applicable for ERM, although some may be questioned e.g. what is a record?}
-McLeod: As the nature of records changes in line with rapid changes in the nature of
information and communication technologies used to create them, so must our strategies for
managing them.
\subsubsection{Introducing: fluent metadata}
\subsection{Protecting authenticity of fluent records}
\subsubsection{Self-archiving: the wiki-way}
\subsection{Protecting reliability of fluent records} 
\subsubsection{Scrolling back through time: record time machines}
\chapter{Conclusions}

%\include{2/aims}						% aims of the project
\include{3/XYZ}			
\include{4/XYZ}	
\include{5/XYZ}
\include{6/XYZ}

\include{7/discussion}               % discussion of results

%\include{8/materials_methods}        % description of lab methods




% --------------------------------------------------------------
%:                  BACK MATTER: appendices, refs,..
% --------------------------------------------------------------

% the back matter: appendix and references close the thesis


%: ----------------------- bibliography ------------------------

% The section below defines how references are listed and formatted
% The default below is 2 columns, small font, complete author names.
% Entries are also linked back to the page number in the text and to external URL if provided in the BibTex file.

% PhDbiblio-url2 = names small caps, title bold & hyperlinked, link to page 
\begin{multicols}{2} % \begin{multicols}{ # columns}[ header text][ space]
\begin{tiny} % tiny(5) < scriptsize(7) < footnotesize(8) < small (9)

\bibliographystyle{Latex/Classes/PhDbiblio-url2} % Title is link if provided
\renewcommand{\bibname}{References} % changes the header; default: Bibliography

\bibliography{9_backmatter/references} % adjust this to fit your BibTex file

\end{tiny}
\end{multicols}

% --------------------------------------------------------------
% Various bibliography styles exit. Replace above style as desired.

% in-text refs: (1) (1; 2)
% ref list: alphabetical; author(s) in small caps; initials last name; page(s)
%\bibliographystyle{Latex/Classes/PhDbiblio-case} % title forced lower case
%\bibliographystyle{Latex/Classes/PhDbiblio-bold} % title as in bibtex but bold
%\bibliographystyle{Latex/Classes/PhDbiblio-url} % bold + www link if provided

%\bibliographystyle{Latex/Classes/jmb} % calls style file jmb.bst
% in-text refs: author (year) without brackets
% ref list: alphabetical; author(s) in normal font; last name, initials; page(s)

%\bibliographystyle{plainnat} % calls style file plainnat.bst
% in-text refs: author (year) without brackets
% (this works with package natbib)


% --------------------------------------------------------------

% according to Dresden med fac summary has to be at the end
%
% Thesis Abstract -----------------------------------------------------


%\begin{abstractslong}    %uncommenting this line, gives a different abstract heading
\begin{abstracts}        %this creates the heading for the abstract page

In this paper current archival institutions are criticized for both their (lack of) response and their way of (not) adapting to an increasingly digital world. Many of the efforts that have been undertaken by current archival institutions are proven to be efforts to enforce 'the paper way of doing things' onto a digitizing world, causing a constant friction between the two. The author of this paper tries to take a leap of faith by establishing a revolutionary vision on the future of archival institutions and the act of archiving in general. Old 'paper views', such as expecting records to be of a singular nature, have a limited (active) life-span, a pre-defined form and a rigid archival bond and context are challenged by introducing new concepts such as 'the never-ending and ever-multiplying record'. By re-inventing, renewing and transforming old concepts into new ones, such as 'fluent metadata', the author finds new ways to protect the authenticity and reliability of these modern records over time, as well as enabling them to be part of a new and spectacular end-user experience, which seems to be left out of the current programs focusing on the development and implementation of E-depots in the Netherlands. 

\end{abstracts}
%\end{abstractlongs}


% ---------------------------------------------------------------------- 


%: Declaration of originality
%\include{9_backmatter/declaration}



\end{document}
