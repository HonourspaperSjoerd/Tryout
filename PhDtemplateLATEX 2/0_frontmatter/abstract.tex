
% Thesis Abstract -----------------------------------------------------


%\begin{abstractslong}    %uncommenting this line, gives a different abstract heading
\begin{abstracts}        %this creates the heading for the abstract page

In this paper current archival institutions are criticized for both their (lack of) response and their way of (not) adapting to an increasingly digital world. Many of the efforts that have been undertaken by current archival institutions are proven to be efforts to enforce 'the paper way of doing things' onto a digitizing world, causing a constant friction between the two. The author of this paper tries to take a leap of faith by establishing a revolutionary vision on the future of archival institutions and the act of archiving in general. Old 'paper views', such as expecting records to be of a singular nature, have a limited (active) life-span, a pre-defined form and a rigid archival bond and context are challenged by introducing new concepts such as 'the never-ending and ever-multiplying record'. By re-inventing, renewing and transforming old concepts into new ones, such as 'fluent metadata', the author finds new ways to protect the authenticity and reliability of these modern records over time, as well as enabling them to be part of a new and spectacular end-user experience, which seems to be left out of the current programs focusing on the development and implementation of E-depots in the Netherlands. 

\end{abstracts}
%\end{abstractlongs}


% ---------------------------------------------------------------------- 
